% Aufgabe 2.1
\section*{Aufgabe 2.1}

\textbf{Aufgabenstellung:} \\
Zeige: $|\vect{x} + \vect{y}| = |\vect{x}| + |\vect{y}|$ gilt genau dann, wenn $|\vect{x}|\vect{y} = |\vect{y}|\vect{x}$ gilt.

\subsection*{Lösungsweg}
Die Gleichung $|\vect{x} + \vect{y}| = |\vect{x}| + |\vect{y}|$ beschreibt den Grenzfall der Dreiecksungleichung (Gleichheit). Wir quadrieren beide Seiten der Gleichung:
\begin{align*}
    |\vect{x} + \vect{y}|^2 &= (|\vect{x}| + |\vect{y}|)^2 \\
    (\vect{x} + \vect{y}) \cdot (\vect{x} + \vect{y}) &= |\vect{x}|^2 + 2|\vect{x}||\vect{y}| + |\vect{y}|^2 \\
    \vect{x} \cdot \vect{x} + 2(\vect{x} \cdot \vect{y}) + \vect{y} \cdot \vect{y} &= |\vect{x}|^2 + 2|\vect{x}||\vect{y}| + |\vect{y}|^2
\end{align*}

Da $\vect{x} \cdot \vect{x} = |\vect{x}|^2$ und $\vect{y} \cdot \vect{y} = |\vect{y}|^2$, vereinfacht sich die Gleichung zu:
\[ 2(\vect{x} \cdot \vect{y}) = 2|\vect{x}||\vect{y}| \quad \Leftrightarrow \quad \vect{x} \cdot \vect{y} = |\vect{x}||\vect{y}| \]

Dies ist der Gleichheitsfall. Diese Gleichheit tritt hierbei ein, weil die Vektoren $\vect{x}$ und $\vect{y}$ linear abhängig sind und in die gleiche Richtung zeigen (also, dass sie kollinear sind).

\indent

Betrachten wir nun die Bedingung $|\vect{x}|\vect{y} = |\vect{y}|\vect{x}$:
\begin{enumerate}
    \item Falls $\vect{x} = \vect{0}$ oder $\vect{y} = \vect{0}$, ist die Gleichung trivialerweise erfüllt ($0=0$).
    \item Falls beide ungleich $\vect{0}$ sind, können wir umstellen:
    \[ \frac{\vect{x}}{|\vect{x}|} = \frac{\vect{y}}{|\vect{y}|} \]
    Dies bedeutet, dass die Einheitsvektoren von $\vect{x}$ und $\vect{y}$ identisch sind. Beide Vektoren zeigen also in exakt dieselbe Richtung.
\end{enumerate}

\indent

Daraus folgt: Wenn die Vektoren in dieselbe Richtung zeigen, ist die Summe ihrer Längen gleich der Länge ihrer Summe, da keine Winkel Komponente die resultierende Länge verkürzt.

% ----------------------
% Aufgabe 2.2
\section*{Aufgabe 2.2}

\textbf{Aufgabenstellung:} \\
Es sei eine Hyperebene im $\mathbb{R}^4$ durch folgende Parameterdarstellung gegeben:
\[ \vect{x} = \sum_{i=1}^{3} \lambda_i \vect{a}_i, \quad \text{mit} \quad \vect{a}_1 = \begin{bmatrix} 1 \\ 8 \\ 2 \\ 0 \end{bmatrix}, \quad \vect{a}_2 = \begin{bmatrix} -1 \\ 2 \\ 7 \\ -2 \end{bmatrix}, \quad \vect{a}_3 = \begin{bmatrix} 10 \\ -1 \\ 3 \\ 4 \end{bmatrix} \]
Geben Sie die Hessesche Normalform an.

\subsection*{Lösungsweg}
Um die Hessesche Normalform $\vect{n} \cdot \vect{x} = 0$ zu finden, bestimmen wir zunächst einen Normalenvektor $\vect{n}'$, der orthogonal auf allen Richtungsvektoren $\vect{a}_i$ steht. Es muss gelten:
\[ \vect{a}_1 \cdot \vect{n}' = 0, \quad \vect{a}_2 \cdot \vect{n}' = 0, \quad \vect{a}_3 \cdot \vect{n}' = 0 \]

Wir setzen $\vect{n}' = [n_1, n_2, n_3, 1]^T$ gemäß dem Hinweis (letzte Komponente = 1). Dies führt zu folgendem linearen Gleichungssystem:
\begin{align*}
    (1) \quad 1n_1 + 8n_2 + 2n_3 + 0(1) &= 0 \\
    (2) \quad -1n_1 + 2n_2 + 7n_3 - 2(1) &= 0 \\
    (3) \quad 10n_1 - 1n_2 + 3n_3 + 4(1) &= 0
\end{align*}

Wir lösen dieses System (z.B. durch Einsetzungsverfahren oder Cramersche Regel):
Aus (1) folgt $n_1 = -8n_2 - 2n_3$. Eingesetzt in (2) und (3):
\begin{align*}
    (2') \quad -(-8n_2 - 2n_3) + 2n_2 + 7n_3 &= 2 \quad \Rightarrow 10n_2 + 9n_3 = 2 \\
    (3') \quad 10(-8n_2 - 2n_3) - 1n_2 + 3n_3 &= -4 \quad \Rightarrow -81n_2 - 17n_3 = -4
\end{align*}

Das daraus resultierende $2 \times 2$ System für $n_2$ und $n_3$ lautet:
\begin{align*}
    (I) \quad 10n_2 + 9n_3 &= 2 \\
    (II) \quad -81n_2 - 17n_3 &= -4
\end{align*}

Um $n_3$ zu eliminieren, multiplizieren wir die Gleichung $(I)$ mit $17$ und $(II)$ mit $9$:
\begin{align*}
    (I') \quad 170n_2 + 153n_3 &= 34 \\
    (II') \quad -729n_2 - 153n_3 &= -36
\end{align*}

Addieren wir nun $(I')$ und $(II')$, erhalten wir:
\begin{align*}
    (170 - 729)n_2 &= 34 - 36 \\
    -559n_2 &= -2 \\
    n_2 &= \frac{2}{559}
\end{align*}

Diesen Wert setzen wir in Gleichung $(I)$ ein, um $n_3$ zu bestimmen:
\begin{align*}
    10 \left( \frac{2}{559} \right) + 9n_3 &= 2 \\
    9n_3 &= 2 - \frac{20}{559} \\
    9n_3 &= \frac{1118 - 20}{559} = \frac{1098}{559} \\
    n_3 &= \frac{1098}{559 \cdot 9} = \frac{122}{559}
\end{align*}

Schließlich berechnen wir $n_1$ unter Verwendung von $n_1 = -8n_2 - 2n_3$:
\begin{align*}
    n_1 &= -8 \left( \frac{2}{559} \right) - 2 \left( \frac{122}{559} \right) \\
    n_1 &= \frac{-16 - 244}{559} = \frac{-260}{559}
\end{align*}

Somit lautet der (noch nicht normierte) Normalenvektor:
\[ \vect{n}' = \frac{1}{559} \begin{bmatrix} -260 \\ 2 \\ 122 \\ 559 \end{bmatrix} \]



\indent

Für die Hessesche Normalform müssen wir $\vect{n}'$ normieren. Die Länge $|\vect{n}'|$ berechnet sich zu:
\[ |\vect{n}'| = \frac{1}{559} \sqrt{(-260)^2 + 2^2 + 122^2 + 559^2} \]
Der finale Vektor für die Hessesche Normalform ist $\vect{n} = \frac{\vect{n}'}{|\vect{n}'|}$, sodass die Ebene durch $\vect{n} \cdot \vect{x} = 0$ beschrieben wird.

\indent

% ------------------------
% Aufgabe 2.3
\section*{Aufgabe 2.3}

\textbf{Aufgabenstellung:} \\
Löse folgendes Gleichungssystem:
\begin{align*}
    3x_1 + 8x_2 - 2x_3 + 2x_4 &= 3 \\
    6x_1 - x_2 + 2x_3 - 3x_4 &= 14 \\
    -2x_1 + 3x_2 + 12x_3 + 5x_4 &= 4 \\
    4x_1 - 5x_2 + 6x_3 - 10x_4 &= 19
\end{align*}

\subsection*{Lösungsweg}
Zunächst führen wir den ersten Schritt des Gauß-Eliminationsverfahrens manuell durch, um die erste Spalte unter dem Pivot-Element $a_{11}=3$ zu eliminieren.

\[
\left(
\begin{array}{cccc|c}
 3 &  8 & -2 &  2 &  3 \\
 6 & -1 &  2 & -3 & 14 \\
-2 &  3 & 12 &  5 &  4 \\
 4 & -5 &  6 & -10 & 19
\end{array}
\right)
\begin{array}{l}
 \\
 R_2 \leftarrow R_2 - 2R_1 \\
 R_3 \leftarrow R_3 + \frac{2}{3}R_1 \\
 R_4 \leftarrow R_4 - \frac{4}{3}R_1
\end{array}
\implies
\left(
\begin{array}{cccc|c}
 3 &  8 & -2 &  2 &  3 \\
 0 & -17 &  6 & -7 &  8 \\
 0 & \frac{25}{3} & \frac{32}{3} & \frac{19}{3} & 6 \\
 0 & -\frac{47}{3} & \frac{26}{3} & -\frac{38}{3} & 15
\end{array}
\right)
\]

\indent

Um die Präzision bei den weiteren Schritten zu gewährleisten und Rechenfehler durch zunehmend komplexe Brüche zu vermeiden, wurde das System zur finalen Lösung numerisch mittels Python (NumPy) gelöst.

\begin{verbatim}
import numpy as np

A_23 = np.array([
    [3, 8, -2, 2],
    [6, -1, 2, -3],
    [-2, 3, 12, 5],
    [4, -5, 6, -10]
])
b = np.array([3, 14, 4, 19])

x = np.linalg.solve(A, b)
print(x)
\end{verbatim}

\textbf{Ergebnis:} \\
Die numerische Lösung liefert die Werte (gerundet auf drei Nachkommastellen):
\begin{align*}
    x_1 &\approx 1,675 \\
    x_2 &\approx 0,165 \\
    x_3 &\approx 0,895 \\
    x_4 &\approx -0,776
\end{align*}

\indent
Die manuelle Überprüfung der ersten Gleichung mit den berechneten Werten zeigt die Korrektheit der Lösung:
\[ 3(1,675) + 8(0,165) - 2(0,895) + 2(-0,776) \approx 5,025 + 1,32 - 1,79 - 1,552 \approx 3,003 \]
Die geringfügige Abweichung resultiert aus den Rundungen der Python-Ausgabe.

% ------------------------
% Aufgabe 2.5
\section*{Aufgabe 2.5}

\textbf{Aufgabenstellung:} \\
(a) Gib die Lösungsmenge des gegebenen $4\times4$-Gleichungssystems an. \\
(b) Gib die Lösungsmenge für den Fall an, dass die -3 auf der rechten Seite unten in -2 verwandelt wird. \\
(c) Streiche aus dem Gleichungssystem in (a) die letzte Zeile heraus, wie auch die Spalte mit
$x_4$. Es entsteht ein dreireihiges quadratisches Gleichungssystem. Skizziere die drei Ebenen, die durch die Gleichungen gegeben sind, in einem räumlichen Koordinatensystem
(man denke sich einen genügend großen gläsernen achsenparallelen Quader gezeichnet
und veranschauliche die Ebenen durch ihre Schnittlinien mit den Quaderwänden). Wie
liegt hier die Lösungsmenge?

\subsection*{Lösungsweg zu (a) und (b)}
Ähnlich wie in Aufgabe 2.3 nutzen wir das Gauß-Verfahren. Ein kritischer Aspekt, der hinzukommt, ist die Untersuchung der Determinante oder des Rangs der Matrix, um festzustellen, ob eine eindeutige Lösung existiert.

Für (b) bedeutet die Änderung der rechten Seite bei gleichbleibender Koeffizientenmatrix eine Verschiebung der Hyperebenen im Raum. Falls das System singulär ist (Rang < 4), kann eine solche Änderung darüber entscheiden, ob das System unendlich viele oder gar keine Lösungen besitzt.

\begin{verbatim}
A_25 = np.array([
    [3, 4, -5, 1],
    [7, -2, 1, -2],
    [1, -10, 11, -4],
    [-4, 6, -6, -3]
])
b_a = np.array([1, -2, -4, -3])
b_b = np.array([1, -2, -4, -2])

# Lösung für (a)
x_a = np.linalg.solve(A_25, b_a)
\end{verbatim}

\subsection*{Lösungsweg zu (c)}
Durch das Entfernen der Variable $x_4$ und der vierten Gleichung betrachten wir nun drei Ebenen im $\mathbb{R}^3$. Geometrisch stellt die Lösungsmenge den Schnittpunkt dieser drei Ebenen dar.


\textbf{Ergebnis:} \\
Die Ebenen schneiden sich in einem Punkt, sofern die Koeffizientenmatrix des reduzierten Systems vollen Rang (Rang 3) hat.