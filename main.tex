\documentclass[12pt,a4paper]{article}

% --- math pkgs ---
\usepackage[utf8]{inputenc}
\usepackage[ngerman]{babel}
\usepackage[T1]{fontenc}
\usepackage{amsmath, amsfonts, amssymb, amsthm}
\usepackage{mathtools}
\usepackage{geometry}
\usepackage{url}
\geometry{a4paper, left=25mm, right=25mm, top=30mm, bottom=30mm}

% --- Präambel for credentails ---
\IfFileExists{credentials.tex}{
    \input{credentials.tex}
}{
    \newcommand{\myuniversity}{[HOCHSCHULE NAME]}
    \newcommand{\myname}{[NAME]}
    \newcommand{\mymatrikel}{[MATRIKELNUMMER]}
}

% --- font ---
\usepackage{lmodern}

% --- header and footer ---
\usepackage{fancyhdr}
\pagestyle{fancy}
\fancyhf{}
\lhead{\textbf{Matrikel-Nr.: \mymatrikel}} 
\chead{}
\rhead{\textbf{\myname}}
\lfoot{\small{Angewandte Mathematik in der Data Science - M.Sc. Data Science}}
\rfoot{Seite \thepage}
\renewcommand{\headrulewidth}{0.4pt}

% --- custom commands (vectors etc.) ---
\newcommand{\R}{\mathbb{R}}
\newcommand{\vect}[1]{\mathbf{#1}}
\newcommand{\mat}[1]{\mathbf{#1}}

% --- title page ---
\title{
    \vspace{-2cm}
    \large \textbf{\myuniversity} \\ 
    \vspace{0.5cm}
    \huge Übungsblatt 1 \\
    \large Modul: Angewandte Mathematik in der Data Science
}
\author{Vorgelegt von: \textbf{\myname (Matrikel-Nr.: \mymatrikel)}}
\date{Abgabetermin: \today}

\begin{document}

\maketitle
\thispagestyle{fancy}

\section*{Einleitung}
In diesem Dokument werden die Lösungen für das erste Übungsblatt dargestellt. Die Aufgaben stammen aus dem Lehrbuch „Höhere Mathematik für Ingenieure - Band 2“. 

\subsubsection*{Technische Notiz}
Dieses Dokument wurde mit \LaTeX{} auf einem lokalen System erstellt. Die Versionsverwaltung und Bereitstellung der Quellcodedateien erfolgt über GitHub, um einen reproduzierbaren Workflow zu gewährleisten.

\textbf{Repository:} 
\url{https://github.com/0xburo/math-for-machine-learning} (private)

\input{aufgaben.tex}
\end{document}